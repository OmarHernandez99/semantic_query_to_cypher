- Se hace una arquitectura
- Se genera data
- Se hace una ui

## Arquitectura
### Modulos
- Graph-Contractor
- Lenguaje-Intermedio
- Pre-Processor
- Post-Processor
- Parser-to-AST
- Query-Maker
- Data-Generation


## Graph-Contractor
El papel del *Graph-Contractor* es el de interactuar con el grafo de conocimiento en cuesti\'on. 

Hace una recopilaci\'on de las etiquetas de los nodos del grafo y de los atributos de estas. 

Adem\'as es responsable de hacer consultas en lenguaje cypher al grafo. 

Lenguaje-Intermedio
Se hace una estructura de un lenguaje intermedio entre el lenguaje natural y el lenguaje de consulta de neo4j, Cypher.

El lenguaje cuenta con su representaci\'on en AST[]

Tiene las caracter\isticas que se creyeron esenciales en un lenguaje de consulta, como los filtros y las agregaciones

Pre-Processor
La tarea del pre-procesador es recibir la query virgen en lenguaje natural y hacerla mas potable para su procesamiento del sistema.

Se hace primero una correci\'on de palabras mal escritas usando distancia Levenshtein[] y el GraphContractor

Se reemplazan las palabras por sus respectivas etiquetas usando el GraphContractor, se usan las etiquetas: "LABEL" y "ATTRIBUTE". 

Esto ayuda sobre todo a que el sistema pueda generalizar mejor y que no sea necesario entrenar el sistema para cada grafo de conocimiento distinto.

Post-Processor
El post-procesador es el encargado de hacer la traducci\'on de la query pre-procesada a un lenguaje intermedio.

Se usa un modelo T5 para hacer la traducci\'on.

Luego se recuperan las palabras originales a las que se les puso etiquetas


Parser-to-AST
El parser recibe una consulta en lenguaje intermedio y la transforma a una instancia de AST.

Query-Maker
El Query-Maker es el encargado de hacer la traducci\'on de la instancia de AST a una consulta en Cypher.

Data-Generation
El m\'odulo de generaci\'on de datos es el encargado de producir consultas sint\'eticas.

Esta compuesto de 3 subm\'odulos:
- Ast-Generator
- Query-Generator
- Word-Module

Ast-Generator
El Ast-Generator es el encargado de generar instancias de AST sint\'eticas. Para eso se apoya en el GraphContractor para saber que nodos y atributos existen en el grafo de conocimiento.

Query-Generator
El Query-Generator es el encargado de generar consultas en lenguaje pseudo-natural. 
A partir de una instancia de AST, genera variaciones del lenguanje que expresan la misma idea, con ayuda del *Word-Module*

Word-Module
El Word-Module es el encargado de proveer palabras que expresen la misma idea, pero que no sean iguales.
Esta implementado de forma que sea independiente del resto del sistema el agregar nuevas etiquetas, palabras, frases o incluso soporte para un nuevo lenguaje.



